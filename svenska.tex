\documentclass{skvitae}
\usepackage{polyglossia}
\setmainlanguage{swedish}
\usepackage{csquotes}
\usepackage{ragged2e}

\author{Simon Sigurdhsson}
\affiliation{}
\street{Engdahlsgatan 6C}
\city{412\,59 Göteborg}
\email{sigurdhsson@gmail.com}
\web{http://sigurdhsson.org}
\phone{}
\github{urdh}
\mobile{0703--29\,30\,54}
\keywords{Simon, Sigurdhsson, Simon Sigurdhsson, CV, Resume}

\begin{document}
	\RaggedRight
	\maketitle

	\section{Utbildning}
	\ind Civilingenjör, Teknisk Matematik, Chalmers Tekniska Högskola, 2008--nuvarande.\\%
		 Masterprogram: \emph{Engineering Mathematics and Computational Science}.\\%
		 Examensarbete: \foreignquote{english}{Solving max-sum problems with the in-the-middle\\heuristic}. %
		 Kandidatarbete: \enquote{Statistisk bildanalys av handgester för människa-dator-interaktion}.

	\section{Uppdrag}
	\ind 2013. Sommarjobb (utveckare inom \emph{Verification efficiency}) hos Volvo Person\-vagnar~AB, Göteborg.
	\ind 2006--2008. Reklamutdelare hos SDR Svensk Direktreklam, Trelleborg.

	\medskip
	\subsection{Ideella uppdrag}
	%\ind 2012--2013. Redaktör hos Fysikteknologsektionens informationsblad, Göteborg.
	\ind 2012--2013. Ekonomiskt ansvarig i Chalmers Studentkårs Film- och Fotocommitté, Göteborg.
	\ind 2011--2014. Fotograf i Chalmers Studentkårs Film- och Fotocommitté, Göteborg.
	\ind 2010--2011. Sekreterare i Fysikteknologsektionens styrelse, Göteborg.
	\ind 2009--2010. Ledamot i Fysikteknologsektionens sexmästeri, Göteborg.
	\ind 2009--2011. Redaktör hos Fysikteknologsektionens informationsblad, Göteborg.

	\section{Kompetenser}
	\subsection{Språk}
	\ind Modersmål: svenska. Talar och skriver även engelska flytande.

	\medskip
	\subsection{IT-kompetenser}
	\ind Goda kunskaper: CSS, C, C++, C\#.NET, HTML, \LaTeX, Mathematica, MATLAB, Microsoft Office, Python, R, Ruby.
	\ind Grundläggande kunskaper: Comsol, Fortran, Java, PHP.

	\medskip
	\subsection{Övrigt}
	\ind Innehar B-körkort sedan 2008.
	\ind Referenser lämnas på begäran.
\end{document}
